% Options for packages loaded elsewhere
\PassOptionsToPackage{unicode}{hyperref}
\PassOptionsToPackage{hyphens}{url}
\PassOptionsToPackage{dvipsnames,svgnames,x11names}{xcolor}
%
\documentclass[
  a4paper,
]{article}

\usepackage{amsmath,amssymb}
\usepackage[]{mathpazo}
\usepackage{iftex}
\ifPDFTeX
  \usepackage[T1]{fontenc}
  \usepackage[utf8]{inputenc}
  \usepackage{textcomp} % provide euro and other symbols
\else % if luatex or xetex
  \usepackage{unicode-math}
  \defaultfontfeatures{Scale=MatchLowercase}
  \defaultfontfeatures[\rmfamily]{Ligatures=TeX,Scale=1}
\fi
% Use upquote if available, for straight quotes in verbatim environments
\IfFileExists{upquote.sty}{\usepackage{upquote}}{}
\IfFileExists{microtype.sty}{% use microtype if available
  \usepackage[]{microtype}
  \UseMicrotypeSet[protrusion]{basicmath} % disable protrusion for tt fonts
}{}
\makeatletter
\@ifundefined{KOMAClassName}{% if non-KOMA class
  \IfFileExists{parskip.sty}{%
    \usepackage{parskip}
  }{% else
    \setlength{\parindent}{0pt}
    \setlength{\parskip}{6pt plus 2pt minus 1pt}}
}{% if KOMA class
  \KOMAoptions{parskip=half}}
\makeatother
\usepackage{xcolor}
\usepackage[margin=25mm]{geometry}
\setlength{\emergencystretch}{3em} % prevent overfull lines
\setcounter{secnumdepth}{-\maxdimen} % remove section numbering
% Make \paragraph and \subparagraph free-standing
\ifx\paragraph\undefined\else
  \let\oldparagraph\paragraph
  \renewcommand{\paragraph}[1]{\oldparagraph{#1}\mbox{}}
\fi
\ifx\subparagraph\undefined\else
  \let\oldsubparagraph\subparagraph
  \renewcommand{\subparagraph}[1]{\oldsubparagraph{#1}\mbox{}}
\fi


\providecommand{\tightlist}{%
  \setlength{\itemsep}{0pt}\setlength{\parskip}{0pt}}\usepackage{longtable,booktabs,array}
\usepackage{calc} % for calculating minipage widths
% Correct order of tables after \paragraph or \subparagraph
\usepackage{etoolbox}
\makeatletter
\patchcmd\longtable{\par}{\if@noskipsec\mbox{}\fi\par}{}{}
\makeatother
% Allow footnotes in longtable head/foot
\IfFileExists{footnotehyper.sty}{\usepackage{footnotehyper}}{\usepackage{footnote}}
\makesavenoteenv{longtable}
\usepackage{graphicx}
\makeatletter
\def\maxwidth{\ifdim\Gin@nat@width>\linewidth\linewidth\else\Gin@nat@width\fi}
\def\maxheight{\ifdim\Gin@nat@height>\textheight\textheight\else\Gin@nat@height\fi}
\makeatother
% Scale images if necessary, so that they will not overflow the page
% margins by default, and it is still possible to overwrite the defaults
% using explicit options in \includegraphics[width, height, ...]{}
\setkeys{Gin}{width=\maxwidth,height=\maxheight,keepaspectratio}
% Set default figure placement to htbp
\makeatletter
\def\fps@figure{htbp}
\makeatother
\newlength{\cslhangindent}
\setlength{\cslhangindent}{1.5em}
\newlength{\csllabelwidth}
\setlength{\csllabelwidth}{3em}
\newlength{\cslentryspacingunit} % times entry-spacing
\setlength{\cslentryspacingunit}{\parskip}
\newenvironment{CSLReferences}[2] % #1 hanging-ident, #2 entry spacing
 {% don't indent paragraphs
  \setlength{\parindent}{0pt}
  % turn on hanging indent if param 1 is 1
  \ifodd #1
  \let\oldpar\par
  \def\par{\hangindent=\cslhangindent\oldpar}
  \fi
  % set entry spacing
  \setlength{\parskip}{#2\cslentryspacingunit}
 }%
 {}
\usepackage{calc}
\newcommand{\CSLBlock}[1]{#1\hfill\break}
\newcommand{\CSLLeftMargin}[1]{\parbox[t]{\csllabelwidth}{#1}}
\newcommand{\CSLRightInline}[1]{\parbox[t]{\linewidth - \csllabelwidth}{#1}\break}
\newcommand{\CSLIndent}[1]{\hspace{\cslhangindent}#1}

\usepackage{mathpazo}

\usepackage{setspace}
\makeatletter
\makeatother
\makeatletter
\makeatother
\makeatletter
\@ifpackageloaded{caption}{}{\usepackage{caption}}
\AtBeginDocument{%
\ifdefined\contentsname
  \renewcommand*\contentsname{Table of contents}
\else
  \newcommand\contentsname{Table of contents}
\fi
\ifdefined\listfigurename
  \renewcommand*\listfigurename{List of Figures}
\else
  \newcommand\listfigurename{List of Figures}
\fi
\ifdefined\listtablename
  \renewcommand*\listtablename{List of Tables}
\else
  \newcommand\listtablename{List of Tables}
\fi
\ifdefined\figurename
  \renewcommand*\figurename{Figure}
\else
  \newcommand\figurename{Figure}
\fi
\ifdefined\tablename
  \renewcommand*\tablename{Table}
\else
  \newcommand\tablename{Table}
\fi
}
\@ifpackageloaded{float}{}{\usepackage{float}}
\floatstyle{ruled}
\@ifundefined{c@chapter}{\newfloat{codelisting}{h}{lop}}{\newfloat{codelisting}{h}{lop}[chapter]}
\floatname{codelisting}{Listing}
\newcommand*\listoflistings{\listof{codelisting}{List of Listings}}
\makeatother
\makeatletter
\@ifpackageloaded{caption}{}{\usepackage{caption}}
\@ifpackageloaded{subcaption}{}{\usepackage{subcaption}}
\makeatother
\makeatletter
\@ifpackageloaded{tcolorbox}{}{\usepackage[many]{tcolorbox}}
\makeatother
\makeatletter
\@ifundefined{shadecolor}{\definecolor{shadecolor}{rgb}{.97, .97, .97}}
\makeatother
\makeatletter
\makeatother
\ifLuaTeX
  \usepackage{selnolig}  % disable illegal ligatures
\fi
\IfFileExists{bookmark.sty}{\usepackage{bookmark}}{\usepackage{hyperref}}
\IfFileExists{xurl.sty}{\usepackage{xurl}}{} % add URL line breaks if available
\urlstyle{same} % disable monospaced font for URLs
\hypersetup{
  pdftitle={A researcher's guide to using the rights retention strategy (version 0.3)},
  pdfauthor={Stephen J Eglen},
  colorlinks=true,
  linkcolor={blue},
  filecolor={Maroon},
  citecolor={Blue},
  urlcolor={Blue},
  pdfcreator={LaTeX via pandoc}}

\title{A researcher's guide to using the rights retention strategy
(version 0.3)}
\author{Stephen J Eglen}
\date{2023-02-09}

\begin{document}
\maketitle
\onehalfspacing

\ifdefined\Shaded\renewenvironment{Shaded}{\begin{tcolorbox}[frame hidden, borderline west={3pt}{0pt}{shadecolor}, boxrule=0pt, interior hidden, breakable, enhanced, sharp corners]}{\end{tcolorbox}}\fi

\hypertarget{what-is-rights-retention}{%
\section{What is rights retention?}\label{what-is-rights-retention}}

The internet has transformed the way that many researchers work. We can
now rapidly share both digital research artefacts and computer
programmes for processing the artefacts. This freedom to share our
digital artefacts is exciting and is rapidly transforming many academic
disciplines. A range of
\href{https://libguides.cam.ac.uk/copyright/researchers}{copyright
licences} now exist to allow researchers to both retain rights on their
work whilst being able to also share their work. The use of standard
licences, particularly the creative commons (CC) family, makes our jobs
as academics relatively painless and robust. This ability to share our
scholarly products is a key academic principle that hopefully most
researchers will find natural and uncontroversial. Furthermore, most UK
grant agencies require data management plans as a condition of funding
to maximise the potential reuse of the research that they fund.

In parallel to this growth of open research, UK research councils and
major funding bodies now require that our published research articles
must freely available without embargo under particular copyright
licences (such as CC-BY). Journal publishers have responded to these
requirements in various ways, e.g.~by offering `gold open access'
routes, that often require payment of substantial article processing
charges (APCs).

Funders also permit `green open access' as a route to compliance with
their rules. This is where the author makes their version of the
manuscript (the peer-reviewed author accepted manuscript, AAM) freely
available via an institutional repository or similar archive. Many
publishers have previously allowed some form of self-archiving, but
usually with constraints, e.g.~imposition of an embargo period (between
6-24 months). These constraints contradict funders' requirements that
prohibit any embargo period.

In response to these restrictive embargo periods, and to encourage open
research, academic institutions and researchers have developed their own
policies to permit sharing of their work. Perhaps the most well-known
example of these is the policy led by
\href{https://osc.hul.harvard.edu/policies/}{Harvard} faculty in 2008.
Within the UK, proposals for a UK Scholarly Communications Licence
\href{https://ukscl.ac.uk/}{UK-SCL} have been created. Internationally,
the
``\href{https://www.coalition-s.org/rights-retention-strategy/}{rights
retention strategy}'\,' been developed by a group of international
funders, cOAlition S, to allow researchers to retain rights on their
scholarly writing and thus meet the requirements of their funders.

This guide describes how authors can retain rights on their manuscripts
using the rights retention strategy.

\hypertarget{why-should-authors-retain-their-rights}{%
\section{Why should authors retain their
rights?}\label{why-should-authors-retain-their-rights}}

Authors should retain their rights so that it they, not a third party,
retain control over uses of their own manuscripts. This includes when
and to whom their manuscript can be disseminated, and means that
researchers treat manuscripts in a similar fashion to other digital
artefacts that they wish to share. Another important reason why
researchers might wish to retain their rights is to comply with funders'
requirements. Ignoring funders' requirements could affect a researcher's
chance of future funding. This might be seen by some researchers as
simple rule-following with little benefit to themselves. However, work
that is available open-access tends to be more highly cited than work
that is only available behind a paywall (Langham-Putrow, Bakker, and
Riegelman 2021). Open-access licences on work allows for rapid reuse by
everyone in the community, including the researchers who originally
created the material. This includes, but is not limited to to books,
talks, wikipedia and social media, for teaching and research.

In many professional disciplines, the creator of the work naturally
maintains rights to their work. Many journal publishers state that they
require exclusive rights to publish the work. This is simply not true
(Suber 2022).

\hypertarget{how-to-retain-rights-to-your-manuscripts}{%
\section{How to retain rights to your
manuscripts}\label{how-to-retain-rights-to-your-manuscripts}}

The process of retaining rights on your manuscripts is simple. Before
submitting your work to a journal, add the following phrase to your
manuscript e.g.~in the Acknowledgements section:

\begin{verbatim}
For the purpose of open access, the author has applied a Creative
Commons Attribution (CC BY) licence to any Author Accepted
Manuscript version arising from this submission.
\end{verbatim}

It is also a good idea to tell the editor, in your cover letter, that
you wish to retain rights to any author-accepted manuscript resulting
from your submission.

That's it. If your paper is accepted by the editor after peer-review,
your author accepted manuscript can then be deposited in a suitable open
repository under a CC-BY licence without embargo. You are then free to
also share the manuscript however you choose. (Some disciplines may
prefer other forms of creative commons licence that prevent
e.g.~commercial use; many funders allow this but it may require you to
apply for an exemption.)

This approach is endorsed and recommended by major research funders,
including the Wellcome Trust and all of the UK Research Councils.

\hypertarget{what-if-things-go-wrong}{%
\subsection{What if things go wrong?}\label{what-if-things-go-wrong}}

In the majority of cases, researchers to date have faced no issues when
using the rights retention strategy to make their manuscripts freely
available without embargo. However, there are several issues to be aware
of:

\begin{enumerate}
\def\labelenumi{\arabic{enumi}.}
\item
  A journal editor may decline reviewing (``desk reject'') your
  manuscript because it uses rights retention language. Whilst
  unfortunate, each publisher is free to decline to review articles for
  many reasons. However, although many traditional publishers have
  voiced their concerns about rights retention
  \url{https://ioppublishing.org/signatories_publish_statement_on_rights_retention_strategy/},
  those same publishers are allowing authors to exert their natural
  rights
  \url{https://github.com/rossmounce/rrs-language-including-outputs}.
\item
  One of the largest publishers, Springer Nature, noted in April 2021
  that in some cases they will effectively ignore rights retention
  language in manuscripts
  \url{https://www.springernature.com/gp/advancing-discovery/blog/blogposts/continuing-the-open-access-transition/19045440}
  and require a transfer of copyright. This will create a conflict once
  the manuscript has been editorially accepted. Ignoring researcher
  rights on their work is highly unethical, and puts researchers in a
  very difficult position. To avoid this stalemate, a researcher could
  seek confirmation from the editor that they can retain their rights
  before submitting their manuscript. A simpler approach would be to
  avoid that publisher.
\item
  If you publish your work in a subscription journal, as long as you
  make your AAM freely available without embargo, you do not need to pay
  the journal an APC. (The journal may still demand other charges, such
  as for colour figures in-print.) However, if you publish your work in
  an open-access journal, you may still need to pay an APC to have your
  work published. Often these costs are covered by institutional
  arrangements with the publisher.
\end{enumerate}

The route to open access can occasionally be more complicated than it
need be. If in doubt, researchers may seek local advice e.g.~from
library staff, when choosing which journal to submit their work to, and
funding sources for any charges. The journal checker tool
\url{https://journalcheckertool.org/} provides information about journal
compliance with funders' OA policies, and provides simple options for
you to follow.

\hypertarget{institutional-support-for-rights-retention}{%
\subsection{Institutional support for rights
retention}\label{institutional-support-for-rights-retention}}

Researchers can and should use the rights retention strategy by
themselves to ensure they retain some control of their own manuscripts,
and to assist them with complying with funders' requirements for open
access. To support their researchers, several UK institutions have
already adopted rights retention policies that apply to their employees.
These policies typically give the institution a non-exclusive right to
archive author accepted manuscripts and make them available via their
repository. The researcher gains an extra layer of institutional support
in the (hopefully rare) case that a publisher raises a concern with
rights retention language in the manuscript (i.e.~an author wishing to
retain rights that are rightfully theirs). Some universities have
publicly stated their support for their researchers should any of their
authors be challenged by a publisher. A list of current UK institutional
policies is available at:
\url{https://github.com/sje30/rrs/blob/main/ukinstitutions.csv} ; this
list is expected to grow considerably during 2023 and 2024. However,
even with an institutional policy supporting the researcher, it is
pragmatic to still include the rights retention statement in your
manuscript.

\hypertarget{closing-comments}{%
\section{Closing comments}\label{closing-comments}}

Adopting the rights retention language in manuscripts is a clear and
simple step that all authors can take to maintain rights over their
work. It has been adopted successfully in the last few years by many
authors. Despite several publishers raising concerns with it in 2021,
there has yet to be any major objection raised when researchers assert
their natural rights. To further support researchers, UK institutions
are adopting policies to allow their employees to share their author
accepted manuscripts without embargo.

\hypertarget{references}{%
\section{References}\label{references}}

\hypertarget{refs}{}
\begin{CSLReferences}{1}{0}
\leavevmode\vadjust pre{\hypertarget{ref-Langham-Putrow2021-qf}{}}%
Langham-Putrow, Allison, Caitlin Bakker, and Amy Riegelman. 2021. {``Is
the Open Access Citation Advantage Real? A Systematic Review of the
Citation of Open Access and Subscription-Based Articles.''} \emph{PLoS
One} 16 (6): e0253129.
\url{https://doi.org/10.1371/journal.pone.0253129}.

\leavevmode\vadjust pre{\hypertarget{ref-Suber2022-dm}{}}%
Suber, Peter. 2022. {``Publishing Without Exclusive Rights.''} \emph{J.
Electron. Publ.} 25 (1). \url{https://doi.org/10.3998/jep.1869}.

\end{CSLReferences}

\hypertarget{changes}{%
\section{Changes}\label{changes}}

\hypertarget{version-0.3}{%
\subsection{version 0.3}\label{version-0.3}}

In this version I thank feedback from the several reviewers. I have
implemented almost all of their suggestions. Of note:

\begin{enumerate}
\def\labelenumi{\arabic{enumi}.}
\tightlist
\item
  The title has been changed.
\item
  A new opening paragraph has been added to highlight the benefits of
  open research, which in turn relies on licencing.
\item
  I have tightened the section `What if things go wrong'?
\end{enumerate}

In addition to these reviewer suggestions, I've added a brief closing
comment.

and now spell-checked!



\end{document}
